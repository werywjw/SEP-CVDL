\subsection*{Author Contributions}
\label{sec:author}
% see https://arxiv.org/pdf/2005.14165.pdf page42

\begin{itemize}
  \item \textbf{Jiawen Wang} preprocessed the image data, aggregated the FER \textsc{GiMeFive}, 
  and implemented different model architectures including \textsc{GiMeFive}, ResNet18, 34, 
  and adapted VGG from scratch (with Python, Shell, Markdown, and Jupyter Notebook), 
  training and evaluation infrastructure, classification score and image labeling script, 
  heat-map plot and Grad-CAM explanation, optimization strategies, 
  together with the corresponding writing part. 
  Also, she is responsible for the math, figures, and tables of this report. 
  Besides, she fixed bugs and typos and improved the code/paper quality. 
  \item \textbf{Leah Kawka} collected and managed databases, 
  setup collaborative infrastructure for databases, created a preprocessing script standardize data sources, 
  was involved in training and writing of preliminary and final report, created presentation video, 
  designed architecture of experimental pipeline, coding of augmentation, Grad-Cam and persistence 
  of training results (pickle), implemented slurm/bash script and evaluating script integrating explainable AI 
  (Grad-Cam, Face recognition, Landmarks) to save a video from given video or live camera stream.
\end{itemize}

\section*{Acknowledgements}

We are deeply grateful to our advisors \textbf{Johannes Fischer} and \textbf{Ming Gui} for their helpful feedback and valuable support during the entire semester. 
We also thank \textbf{Prof. Dr. Björn Ommer} for providing this interesting practical course. 

A special acknowledgement to our former team members \textbf{Mahdi Mohammadi}, who joined us till the end of the 
second phase of the project and shared research in conclusion, data pre-processing, and CAM-Images inquiry, 
and \textbf{Tanja Jaschkowitz}, who joined us during the first phase of the project and shared two notebook scripts.
  
% The large face is so funny, not sure where we should put it
\begin{figure}[ht]
  \centering
   \includegraphics[width=\linewidth]{visual.png}
   \caption{A friendly smile based on the face landmark.} 
   \label{fig:visual}
\end{figure}

\newpage

% \begin{itemize}
%   \item We collected datasets, preprocessed images or videos, 
%   and evaluated the training and testing data thoroughly from various public databases listed in.
%   \item To prepare training and testing images for loading we wrote a script.
%   \item All implemented classification models, such as VGG16, ResNet, 
%   or \textsc{GiMeFive} we built from scratch, 
%   and compared in accuracy in consideration of Layers and Architecture.
%   \item Subsequently optimization was executed with several techniques in a systematic manner. 
%   \item The classification scores of each emotion class, 
%   as well as the parameters to perform the confusion matrix heat map, 
%   are saved (in a serialised PyTorch state dictionary,) with respect to each image or video frame. 
%   \item We provide qualitative benefits such as interpretability to explain our model with Grad-CAM, 
%   combined with Face Landmarks 
%   Landmarks and the indication of the classified emotion percentages, 
%   emphasizing the main emotion.
%   \item To evaluate a video from a given video or live camera we created a script. 
% 	\item To illustrate the real-world performance of our best model \textsc{GiMeFive}, 
%   we edited and provide a demo video.
% \end{itemize}