\section{Introduction}
\label{sec:intro}

% rewrite
\textit{Facial emotion recognition} (FER)~\cite{Ko18,JainSS19} is a topic of significant frontier and ongoing debate, 
not only in our daily lives but also in the fields of \textit{artificial intelligence} (AI).
In this report, we aim to leverage several deep 
\textit{convolutional neural networks} (CNNs) to detect and interpret six basic universally recognized and expressed human facial emotions 
(i.e., happiness, surprise, sadness, anger, disgust, and fear). 
To make our model more transparent, 
we explain this emotion classification task with \textit{gradient-weighted class activation mapping} (Grad-CAM). 

% update later
Our main contributions can be summarized as follows.
\begin{itemize}
  \item We collect, preprocess, and evaluate the training and testing data throughly, 
  (both image and video) from various public databases. 
  \item We implement all classification models from scratch and optimize them with several techniques in a systematical manner. 
  Meanwhile, we provide the classification scores of each emotion class in a detailed script with respect to each image. 
  \item We give the video demo to illustrate the real-world performance of our best model.
  \item We provide qualitative benefits such as interpretability to explain our model with Grad-CAM. 
\end{itemize}

An overview of the experimental pipeline of our project is illustrated in \Cref{fig:pipeline}. 
The structure of rest of the report is arranged as follows. 
\Cref{sec:related} contains the related work of our research. 
In \Cref{sec:setup}, 
we address the datasets we collected and the model architecture we implemented. 
The evaluation results of our models are given in \Cref{sec:evaluation} with interpretability. 
\Cref{sec:optim} describes the optimization strategies such as data augmentation and hyper-parameter tuning. 
We provide the conclusion and discussion in \Cref{sec:conclusion}. 

\begin{figure*}[ht]
  \centering
   \includegraphics[width=\linewidth]{pipeline.png}
   \caption{Overview the experimental pipeline of our project} 
   \label{fig:pipeline}
\end{figure*}

% add demo: see https://github.com/werywjw/SEP-CVDL/blob/main/paper/Selvaraju_Cogswell_Grad-CAM.pdf
\section{Related Work}
\label{sec:related}

\paragraph{Interpretable FER.}

\citet{YinTLS019} focus on a specific area of interpretable visual recognition by learning from data a structured facial representation. 
\citet{Malik0R21} 

\paragraph{Explainable AI.}
To understand the decision-making process of our model, 
we aim to explain our model in a more transparent and interpretable way using the Grad-CAM, 
i.e., Gradient-weighted CAM~\cite{SelvarajuCDVPB17}, 
a technique that is easier to implement with different architectures. 

Class Activation Mapping (CAM) is a technique popularly used in CNNs to visualize and understand the regions of an input image that contribute the most to a particular class prediction. 
Generally speaking, 
CAM~\cite{ZhouKLOT16} helps interpret CNN decisions by providing visual cues about the regions that influenced the classification, 
as it highlights the important regions of an image or a video, 
aiding in the understanding of the behavior of the model, 
which is especially useful for model debugging and further improvement. 
Typically,
CAM is applied to the final convolutional layer of a CNN. 
Besides proposing a method to visualize the discriminative regions of a CNN trained for the classification task, 
we adopt this approach from \citet{ZhouKLOT16} to localize objects without providing the model with any bounding box annotations. 
The model can therefore learn the classification task with class labels and is then able to localize the object of a specific class in an image or video. 

%~\footnote{~\url{https://medium.com/@stepanulyanin/implementing-grad-cam-in-pytorch-ea0937c31e82}}
% The weights connecting the feature maps to the output class are obtained.
% The weighted combination of feature maps, 
% representing the importance of each spatial location, is used to generate the CAM heatmap.
Despite CAM can provide valuable insights into the decision-making process of deep learning models, especially CNNs, 
CAM must be implemented in the last layer of a CNN or before the fully connected layer.
% We implement using the libraries such as PyTorch and OpenCV~\footnote{~\url{https://opencv.org}}.

\citet{chattopadhay2018grad} proposed Grad-CAM++,

\section{Experimental Setup}
\label{sec:setup}

All the experiments are implemented in Python and Shell for generating scripts. 

\subsection{Dataset Description}
\label{sec:setup:datasets}

% see 
% https://arxiv.org/pdf/2303.15889.pdf
% https://www.mdpi.com/1424-8220/22/21/8089

\begin{table*}[ht]
  \centering
  \resizebox{.99\textwidth}{!}{
  \begin{tabular}{@{}lccc@{}}
    \toprule
    \textsc{Dataset} & \#Images in training set & \#Images in testing set & \# Video/Images in validation set  \\
    \midrule
    Denver Intensity of Spontaneous Facial Action Database (DISFA~\cite{mavadati2013disfa}) & - & 27 & - \\
    \midrule
    Real-world Affective Faces Database (RAF-DB~\cite{li_reliable_2017,li2019reliable}) & 9747 & 2388 & 600 \\
    Facial Expression Recognition 2013 (FER2013~\cite{BarsoumZCZ16}) & 23743 & 5945 & 600  \\
    FER AffectNet Database (FER AffectNet~\cite{Mollah2019ANet}) & 21045 &-& -  \\
    Extended Cohn-Kanade Dataset Plus (CK+~\cite{LuceyCKSAM10}) & 309 &-& - \\
    Taiwanese Facial Expression Image Database (TFEID~\cite{tfeid,LiGL22}) & 229 &-& - \\
    \midrule
    \midrule
    FER \textsc{GiMeFive} & \textbf{55073} & \textbf{8333} & \textbf{600} \\
    \bottomrule
  \end{tabular}
  }
  \caption{Overview of statistics of datasets used in our experiment 
  (Note that 600 images in the validation set above is given from our advisors, 
  we provide the detailed training image statistics for each emotion class in \Cref{tab:emotion})}
  \label{tab:data}
\end{table*}

\begin{table}[ht]
  \centering
  \resizebox{.47\textwidth}{!}{
  \begin{tabular}{@{}lcccccc@{}}
      \toprule
      \textsc{Dataset (Train)} & \#happiness & \#surprise & \#sadness & \#anger & \#disgust & \#fear \\
      \midrule
      RAF-DB & 4772 & 1290 & 1982 & 705 & 717 & 281 \\
      FER2013 & 7215 & 3171 & 4830 & 3994 & 436 & 4097 \\
      FER AffectNet & 3091 & 4039 & 5044 & 3218 & 2477 & 3176 \\
      CK+ & 69 & 83 & 28 & 45 & 59 & 25 \\
      TFEID & 40 & 36 & 39 & 34 & 40 & 40 \\
      \midrule
      \midrule
      FER \textsc{GiMeFive} (Train)  & 15187 & 8619 & 11923 & 7996 & 3729 & 7619 \\
      \midrule
      FER \textsc{GiMeFive} (Test) & 2959 & 1160 & 1725 & 1120 & 271 & 1098 \\
      FER \textsc{GiMeFive} (Valid) & 100 & 100 & 100 & 100 & 100 & 100 \\ 
      \bottomrule
  \end{tabular}
  }
\caption{Overview of the training image statistics for each emotion class}
\label{tab:emotion}
\end{table}

To initiate the project, % training, 
we acquired the image databases such as RAF-DB~\cite{li_reliable_2017,li2019reliable}, 
FER2013~\cite{BarsoumZCZ16}, FER AffectNet~\cite{Mollah2019ANet}, 
CK+~\cite{LuceyCKSAM10}, and TFEID~\cite{tfeid,LiGL22}, 
as well as the video database DISFA~\cite{MavadatiMBTC13} from public institutions and kaggle~\cite{kaggle_rafdb,kagaff}
Based on these databases, we created a dataset by augmentation to increase the variety, 
and full details of augmentation (see \Cref{sec:optim:aug}). 
In terms of illustrating the content of used pictures, we exclusively analyze human faces representing 6 emotions. 
That is, 
we generalized a folder structure annotating the labels 0 (happiness), 1 (surprise), 2 (sadness), 3 (anger), 4 (disgust), and 5 (fear). 
Besides the original format of images and videos, we set standards for extracting frames from the videos, 
resizing training pictures to $64\times 64$ pixels, and saving them in the JPG format.

The images are converted to greyscale with three channels, 
as our original CNN is designed to work with three-channel inputs with random rotation and crop. 
Emotions were assigned tags to each individual picture in a CSV file to facilitate further processing in the model.
We create a custom dataset, which is a collection of data relating to all training images we collected, 
using PyTorch.

\subsection{Model Architecture}
\label{sec:setup:model}

\tikzstyle{block} = [rectangle, draw, fill=LMUGreen!30, text width=5em, text centered, rounded corners, minimum height=4em]
\tikzstyle{line} = [draw, -latex', line width=0.7pt]

\begin{figure*}[ht]
  \centering
  \resizebox{.99\textwidth}{!}{
  \begin{tikzpicture}[node distance=2.7cm, auto]

      \node [block] (input) {Input Layer};
      \node [block, right of=input] (conv1) {Conv Block 1 \\ 64 Filters};
      \node [block, right of=conv1] (conv2) {Conv Block 2 \\ 128 Filters};
      \node [block, right of=conv2] (conv3) {Conv Block 3 \\ 256 Filters};
      \node [block, right of=conv3] (conv4) {Conv Block 4 \\ 512 Filters};
      \node [block, right of=conv4] (conv5) {Conv Block 5 \\ 1024 Filters};
      \node [block, right of=conv5] (adaptivePool) {Adaptive Avg Pool};
      \node [block, right of=adaptivePool] (fc) {Fully Connected Block};
      \node [block, right of=fc] (softmax) {Softmax Layer};

      \path [line] (input) -- (conv1);
      \path [line] (conv1) -- (conv2);
      \path [line] (conv2) -- (conv3);
      \path [line] (conv3) -- (conv4);
      \path [line] (conv4) -- (conv5);
      \path [line] (conv5) -- (adaptivePool);
      \path [line] (adaptivePool) -- (fc);
      \path [line] (fc) -- (softmax);

  \end{tikzpicture}
  }
  \caption{Overview of the model architecture (see \Cref{fig:modeldetail} for a detailed version)} 
  \label{fig:model}
\end{figure*}

\Cref{fig:model} illustrates the overview of our model architecture. 
The input of our emotion recognition model is an image with 3 channels at $64 \times 64 $ resolution. 
The output is 6 emotion classes: happiness, surprise, sadness, anger, disgust, and fear. 
We implement an emotion classification model from scratch with four convolution blocks at the very beginning. 
Despite larger kernel can provide more information and wider area view due to more parameters, 
we use a $3 \times 3$ kernel size for all convolutional layers, 
as it is efficient to train and shares the weights without expensive computation. 
Following each convolutional layer, 
batch normalization is used for stabilizing the learning by normalizing the input to each layer. 
We interleaved with the max pooling layer because it reduces the spatial dimensions of the input volume. 
Afterward, three linear layers are applied to extract features to the final output. 
We also add a 50\% dropout layer to prevent overfitting. 
Verified by \citet{BarsoumZCZ16}, 
the dropout layers are effective in avoiding model overfitting. 
The activation function after each layer is \textit{Rectified Linear Unit} (ReLU), 
since it introduces the non-linearity into the model, 
allowing it to learn more complex patterns. 

In order to find the best hyperparameter configuration (see \Cref{tab:hyper} for details) of the model, 
we utilize the parameter grid from Sklearn.
% ~\footnote{\url{https://scikit-learn.org/stable/modules/generated/sklearn.model_selection.ParameterGrid.html}}.
Additionally, we increase the depth of the network by adding some convolutional layers to learn more complex features. 
To help the training of deeper networks more efficiently, 
we add the residual connections, 
as they allow gradients to flow through the network more easily, improving the training for deep architectures. 
Moreover, 
we add \textit{squeeze and excitation} (SE) blocks to apply channel-wise attention. 

\section{Evaluation}
\label{sec:evaluation}

For evaluation, we use the metric accuracy. 
We report all the training, testing, and validation accuracy in \% to compare the performance of our models. 
The loss function employed for all models is cross-entropy (CE), which is typically for multi-class classification. 
That is:
\begin{equation}
  \mathcal{L}_{\text{CE}} = -\sum_{i=1}^{n} y_i \log(p_i),
\end{equation}
where $y_i$ is the true label and $p_i$ is the predicted probability of the $i$-th class.


\subsection{Evaluation Results}
\label{sec:evaluation:results}

\begin{table*}[ht]
  \centering
  % \resizebox{.99\textwidth}{!}{
  \begin{tabular}{@{}llccccr@{}}
    \toprule 
    \multirow{2}{*}{\textsc{Dataset}}&\multirow{2}{*}{\textsc{Models}}&\multirow{2}{*}{\textsc{Architecture}} & \multicolumn{3}{c}{\textsc{Accuracies}} & \multirow{2}{*}{\textsc{\# Parameters}} \\
    \cline{4-6}
    &&& Train & Test & Valid  &  \\
    \midrule
    \multirow{6}{*}{RAF-DB~\cite{li_reliable_2017,li2019reliable}}     & ResNet18~\cite{HeZRS16} & Residual Block  & \textbf{98.9} & 81.3 & 67.9 & 11179590 \\
    &Ours (13 layers) & +BN-SE & 96.6 & 80.6 & 66.8 & 2606086 \\ 
    &Ours (10 layers) & -BN-SE & 96.3 & 76.9 & 60.6 & 10474118 \\
    &Ours (16 layers) & +BN+SE & 98.4 & 81.7 & 71.1 & 10478598 \\
    &Ours (15 layers) & +BN-SE & 98.6 & \textbf{83.1} & \textbf{72.1} & 10478086 \\
    &Ours (17 layers) & +BN-SE & 97.5 & 82.5 & 70.0 & 41950726 \\ 
    \midrule
    \multirow{3}{*}{FER2013~\cite{BarsoumZCZ16}} & Ours (13 layers) & +BN-SE & 86.6 & 64.1 & 40.2 & 2606086 \\
    &Ours (15 layers) & +BN-SE & 89.6 & \textbf{65.6} & 40.7 & 10478086 \\
    &Ours (17 layers) & +BN-SE & \textbf{96.0} & 65.5 & \textbf{41.6} & 41950726 \\
    \bottomrule
  \end{tabular}
  % }
  \caption{Accuracies (\%) for different models (with specific architectures and numbers of parameters) in our experiments 
  (Note that SE stands for the squeeze and excitation block and BN for the batch normalization; 
  +/- represent with/without respectively)}
  \label{tab:model}
\end{table*}

Adding an extra convolutional block to the model with more parameters does not necessarily lead to better performance.
Batch normalization can indeed improve the performance of the model. 

\Cref{tab:model} shows the test result aggregated from the database RAF-DB~\cite{kaggle_rafdb} and FER2013~\cite{kaggle_fer}.
% ~\footnote{\url{https://www.kaggle.com/datasets/shuvoalok/raf-db-dataset}}. 
Different combinations of functions from the \texttt{pytorch.transforms} library are tested for augmentation from those already established filters. % that have been developed. 
As seen in \Cref{tab:model}, 
our CNN without random augmentation outperforms the other models in terms of accuracy, 
indicating that this kind of augmentation is not able to help our model predict the correct label, 
thus we later aim to optimize with other augmentation techniques to capture more representative features of different emotions.
Further research is orientated on papers engaging similar investigations~\cite{ZeilerF14,li_reliable_2017,VermaMRMV23}.


\subsection{Interpretable Results}
\label{sec:evaluation:inter}

\paragraph{Classification Scores.}
To further analyze the separate scores of each class of the model, 
we write a script that takes a folder path as input and iterates through the images inside a subfolder to record the performance of the model with respect to each emotion class. 
This CSV file is represented with the corresponding classification scores. 

\begin{table*}[ht]
  \centering
  % \begin{tabular}{l|c|c|c|c|c|c}
  % \begin{tabular}{lcccccc} 
  \begin{tabular}{@{}lcccccc@{}}
  \toprule
  filepath & happiness & surprise & sadness & anger & disgust & fear \\
  \midrule
  archive/RAF-DB/test/test\_0524\_aligned\_happiness.jpg & \textbf{1.00} & 0.00 & 0.00 & 0.00 & 0.00 & 0.00 \\
  archive/RAF-DB/test/test\_0093\_aligned\_happiness.jpg & \textbf{1.00} & 0.00 & 0.00 & 0.00 & 0.00 & 0.00 \\
  archive/RAF-DB/test/test\_2193\_aligned\_sadness.jpg & 0.03 & 0.01 & \textbf{0.90} & 0.01 & 0.02 & 0.04 \\
  archive/RAF-DB/test/test\_1214\_aligned\_happiness.jpg & \textbf{1.00} & 0.00 & 0.00 & 0.00 & 0.00 & 0.00 \\
  archive/RAF-DB/test/test\_1816\_aligned\_surprise.jpg & 0.00 & \textbf{1.00} & 0.00 & 0.00 & 0.00 & 0.00 \\
  archive/RAF-DB/test/test\_0294\_aligned\_surprise.jpg & 0.01 & \textbf{0.99} & 0.00 & 0.00 & 0.00 & 0.00 \\
  archive/RAF-DB/test/test\_1128\_aligned\_happiness.jpg & \textbf{1.00} & 0.00 & 0.00 & 0.00 & 0.00 & 0.00 \\
  archive/RAF-DB/test/test\_1799\_aligned\_sadness.jpg & 0.37 & 0.02 & \textbf{0.45} & 0.02 & 0.1 & 0.03 \\
  archive/RAF-DB/test/test\_0610\_aligned\_sadness.jpg & 0.02 & 0.00 & \textbf{0.74} & 0.02 & 0.21 & 0.01 \\
  archive/RAF-DB/test/test\_1373\_aligned\_anger.jpg & 0.00 & 0.00 & 0.00 & \textbf{1.00} & 0.00 & 0.00 \\
  archive/RAF-DB/test/test\_1788\_aligned\_fear.jpg & 0.00 & 0.00 & 0.00 & 0.00 & 0.00 & \textbf{1.00} \\
  archive/RAF-DB/test/test\_0007\_aligned\_disgust.jpg & 0.03 & 0.05 & 0.03 & \textbf{0.58} & 0.18 & 0.13 \\
  archive/RAF-DB/test/test\_0804\_aligned\_disgust.jpg & 0.02 & 0.00 & 0.18 & 0.02 & \textbf{0.77} & 0.00 \\
  % \vdots & \vdots & \vdots & \vdots & \vdots & \vdots & \vdots \\
  \bottomrule
  \end{tabular}
  \caption{Overview of our random testing results examples extracted from the CSV file}
  \label{tab:testcsv}
\end{table*}  

In our case, we leverage the fifth convolutional layer of our model to generate the CAM heatmap.
The \textit{global average pooling} (GAP) layer, 
which computes the average value of each feature map to obtain a spatial average of feature maps, 
is used to obtain a spatial average of the feature maps. 

\section{Optimization Strategies}
\label{sec:optim}

To further understand and enhance the performance of the model during training, 
% we focus on the following tasks, 
% i.e., data augumentation (see \Cref{sec:optim:aug}), 
% CAM (see \Cref{sec:optim:cam}), and Grad-CAM (see \Cref{sec:optim:gcam}). 
%~\footnote{\url{https://github.com/maelfabien/Multimodal-Emotion-Recognition}}. 

\subsection{Data Augmentation}
\label{sec:optim:aug}

In deep learning and AI, 
augmentation stands as a transformative technique, 
empowering algorithms to learn from and adapt to a wider range of data. 
By introducing subtle modifications to existing data points, 
augmentation effectively expands the dataset, 
enabling models to generalize better and achieve enhanced performance. 
As models encounter slightly altered versions of familiar data, 
they are forced to make more nuanced and robust predictions. 
With this process, we aim to prevent overfitting, which is a common pitfall in machine learning. 
Additionally, we guide the training process to enhance the recognition and handling of real-world variations.
Meanwhile, we create various replications of existing photos by randomly altering different properties such as size, brightness, color channels, or perspectives.

\subsection{Hyper-parameter Tuning}
\label{sec:optim:tuning}

\begin{table}[ht]
  \centering
  \begin{tabular}{@{}lc@{}}
    \toprule
    \textsc{Hyperparameter} & \textsc{Value} \\
    \midrule
    Learning rate & \{0.01, 0.001, 0.0001\}  \\
    Batch size & \{8, 16, 32, 64\} \\
    Dropout rate & \{0.2, 0.5\} \\
    Convolution depth & \{4, 5, 6\} \\
    Fully connected depth & \{2, 3\} \\
    Batch normalization & \{\texttt{True}, \texttt{False}\} \\
    Pooling & \{\texttt{max}, \texttt{adaptive avg}\} \\
    Optimizer & \{\texttt{Adam}, \texttt{AdamW}, \texttt{SGD}\} \\
    Activation & \{ \texttt{relu}, \texttt{tanh}, \texttt{elu}, \texttt{gelu}\} \\
    Epoch & \{40, 80\} \\
    Early stopping & \{\texttt{True}, \texttt{False}\} \\
    Patience & \{5, 10\} \\
    \bottomrule
  \end{tabular}
  \caption{Explored hyperparameter space for our models}
  \label{tab:hyper}
\end{table}

\section{Conclusion and Discussion}
\label{sec:conclusion}

\section{Limitation}
\label{sec:limitation}

\newpage*
% \clearpage