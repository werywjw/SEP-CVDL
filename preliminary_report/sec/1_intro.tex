\section{Introduction}
\label{sec:intro}


Provide an overview of the different modalities used for emotion detection, such as facial expressions, speech, text, and physiological signals.
Highlight the challenges and limitations of existing emotion detection techniques.

\Cref{fig:schedule} depicts an overview of our time schedule for the final project. 

The structure of this report is arranged as follows. 
\Cref{sec:research} contains the related work of our research. 
\Cref{sec:approach} provides the datasets we used, 
the model architecture, 
and preliminary evaluation results of our model. 

%-------------------------------------------------------------------------
% All authors will benefit from reading Mermin's description of how to write mathematics:
% \url{http://www.pamitc.org/documents/mermin.pdf}.

% Short captions should be centered.

% \begin{figure}[t]
%   \centering
%   \fbox{\rule{0pt}{2in} \rule{0.9\linewidth}{0pt}}
%    %\includegraphics[width=0.8\linewidth]{egfigure.eps}

%    \caption{Example of caption.
%    It is set in Roman so that mathematics (always set in Roman: $B \sin A = A \sin B$) may be included without an ugly clash.}
%    \label{fig:onecol}
% \end{figure}

% \begin{figure*}
%   \centering
%   \begin{subfigure}{0.68\linewidth}
%     \fbox{\rule{0pt}{2in} \rule{.9\linewidth}{0pt}}
%     \caption{An example of a subfigure.}
%     \label{fig:short-a}
%   \end{subfigure}
%   \hfill
%   \begin{subfigure}{0.28\linewidth}
%     \fbox{\rule{0pt}{2in} \rule{.9\linewidth}{0pt}}
%     \caption{Another example of a subfigure.}
%     \label{fig:short-b}
%   \end{subfigure}
%   \caption{Example of a short caption, which should be centered.}
%   \label{fig:short}
% \end{figure*}

