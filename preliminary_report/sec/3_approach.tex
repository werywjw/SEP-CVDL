\section{Approach}
\label{sec:approach}

\subsection{Dataset Aquisition and Processing}
Firstly, 
for all the image data from the training dataset~\cite{li2017reliable,li2019reliable}, 
we filter out neutral instances from the original dataset, 
the emotion labels are denoted as 1 (Surprised), 2 (Fearful), 3 (Disgusted), 4 (Happy), 5 (Sad), and 6 (Angry) for simplicity. 
Afterward, 
we transform and resize the images to \texttt{(64,64)}. 

Describe the data collection methods employed, including the datasets used, the acquisition procedures, and the annotation standards.

\subsubsection{Pictures and Labeling}
\subsubsection{Video Processing}
\subsubsection{Augmentation}

\subsection{Model Implementation}

\subsubsection{Training Infrastructure}

\subsubsection{Model Architecture}
We implemented an emotion-classification model with 3 convolution layers.


\subsubsection{Model Training and Validation}

We add a \texttt{dropout} layer to prevent overfitting. 
In order to find the best hyperparameter configuration (see \cref{tab:hyper} for details) of the model, 
we utilize the parameter grid from sklearn~\footnote{\url{https://scikit-learn.org/stable/modules/generated/sklearn.model_selection.ParameterGrid.html}}. 

\begin{table}[ht]
    \centering
    \begin{tabular}{@{}lc@{}}
      \toprule
      Hyperparameter & Configuration \\
      \midrule
      Learning rate & \{0.1, 0.01, 0.001, 0.0001\}  \\
      Batch size & \{8, 16, 32, 64\} \\
      Dropout rate & \{0.5\} \\
      Epoch & \{20\} \\
      Early stopping & \{\texttt{True}, \texttt{False}\} \\
      Patience & \{5\} \\
      \bottomrule
    \end{tabular}
    \caption{Explored hyperparameter space for our model}
    \label{tab:hyper}
  \end{table}

For evaluation, we use the metric accuracy.

Explain the emotion detection techniques adopted, including the algorithms, models, and machine learning approaches used.
Model training with GPU (Tanja, Leah, Jiawen, Mahdi, 15.Jan)
Model testing acc (Jiawen, 15.Jan)
Elaborate on the experimental setup and evaluation criteria, including the performance metrics used to assess the accuracy and effectiveness of the system.

\subsection{Explainable AI}
\subsubsection{CAM-Images (Mahdi)}
\subsubsection{Heatmap (Mahdi, Leah)}
\subsubsection{CAM-Videos (Tanja)}
\subsubsection{Video Green-Square (Leah)}
\subsubsection{Predictions .csv (Tanja)}