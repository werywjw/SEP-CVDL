\section{Introduction}
\label{sec:intro}

%-------------------------------------------------------------------------
\subsection{Language}

All manuscripts must be in English.

\subsection{Paper length}
The final paper, excluding the references section, must be no longer than eight pages in length. The preliminary paper, excluding the references section, should be around two pages in length.
The references section will not be included in the page count, and there is no limit on the length of the references section.
For example, a paper of eight pages with two pages of references would have a total length of 10 pages.

Overlength papers will simply not be reviewed.
\subsection{Mathematics}

Please number all of your sections and displayed equations as in these examples:
\begin{equation}
  E = m\cdot c^2
  \label{eq:important}
\end{equation}
and
\begin{equation}
  v = a\cdot t,
  \label{eq:also-important}
\end{equation}
or something like $e^{(i\pi)} + 1 = 0$.
It is important for readers to be able to refer to any particular equation.
Just because you did not refer to it in the text does not mean some future reader might not need to refer to it.
It is cumbersome to have to use circumlocutions like ``the equation second from the top of page 3 column 1''.
All authors will benefit from reading Mermin's description of how to write mathematics:
\url{http://www.pamitc.org/documents/mermin.pdf}.

\subsection{Figures}

\begin{figure}[t]
  \centering
  \fbox{\rule{0pt}{2in} \rule{0.9\linewidth}{0pt}}
   %\includegraphics[width=0.8\linewidth]{egfigure.eps}

   \caption{Example of caption.
   It is set in Roman so that mathematics (always set in Roman: $B \sin A = A \sin B$) may be included without an ugly clash.}
   \label{fig:onecol}
\end{figure}

\begin{figure*}
  \resizebox{\textwidth}{!}{
    \begin{ganttchart}[
      % hgrid,
      vgrid,
      time slot format=isodate,
      x unit=0.3cm, 
      y unit title=1.1cm,
      y unit chart=0.6cm,
      bar height=0.5,
      bar top shift=0.2,
      group right shift=0,
      group top shift=0.1,
      group height=.3,
      group peaks width={0.2},
      milestone left shift=0.5,
      milestone right shift=0.5,
      title label font=\bfseries\Large
  ]{2023-12-21}{2024-02-23}
      \gantttitlecalendar{year, month} \\
      \ganttgroup[bar/.append style={fill=LightPurple}]{Preliminary Report}{2023-12-21}{2024-01-17} \\
      \ganttbar[bar/.append style={fill=LightPurple}]{Data Search \& Preprocessing}{2023-12-21}{2023-12-24} \\
      \ganttbar[bar/.append style={fill=LightPurple}]{Model Training \& Validating}{2023-12-25}{2024-01-04} \\
      \ganttbar[bar/.append style={fill=LightPurple}]{Model Selection \& Fine-Tuning}{2024-01-04}{2024-01-10} \\
      \ganttbar[bar/.append style={fill=LightPurple}]{Report Writing}{2024-01-05}{2024-01-16} \\
      \ganttmilestone{Deadline 1}{2024-01-18} \\
      \ganttgroup[bar/.append style={fill=LMUGreen}]{Presentation}{2024-01-04}{2024-02-07} \\
      \ganttbar[bar/.append style={fill=LMUGreen}]{Explainable AI}{2024-01-04}{2024-01-18} \\
      \ganttmilestone{Deadline 2}{2024-02-08} \\
      \ganttgroup[bar/.append style={fill=cvprblue}]{Final Submission}{2024-01-19}{2024-02-22} \\
      \ganttbar[bar/.append style={fill=cvprblue}]{Report Writing}{2024-01-17}{2024-02-22} \\
      \ganttmilestone{Deadline 3}{2024-02-23}
      \ganttlink{elem1}{elem2}
      \ganttlink{elem2}{elem3}
      \ganttlink{elem2}{elem4}
      \ganttlink{elem4}{elem5}
  \end{ganttchart}
  }
  \caption{Overview of the time schedule on the final project}
  \label{fig:schedule}
\end{figure*}

\begin{figure*}
  \centering
  \begin{subfigure}{0.68\linewidth}
    \fbox{\rule{0pt}{2in} \rule{.9\linewidth}{0pt}}
    \caption{An example of a subfigure.}
    \label{fig:short-a}
  \end{subfigure}
  \hfill
  \begin{subfigure}{0.28\linewidth}
    \fbox{\rule{0pt}{2in} \rule{.9\linewidth}{0pt}}
    \caption{Another example of a subfigure.}
    \label{fig:short-b}
  \end{subfigure}
  \caption{Example of a short caption, which should be centered.}
  \label{fig:short}
\end{figure*}

